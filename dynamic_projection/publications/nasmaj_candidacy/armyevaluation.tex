\section{ARmy Design and Interaction User Study}
This section is from a paper that is appearing in Procams in 2012\cite{army}.  As part of the requirements to complete a masters degree at RPI, Andrew Dolce developed an ARmy simulation application for the tabletop system.  We submitted a paper which presented the system as well as a user study comparing the game to a non-augmented equivalent.  I took the lead on designing and evaluating the study.
%We had several goals in designing the user study for our
%spatially augmented gaming prototype.  
\subsection{ARmy Game}

The ARmy application was a game in which plastic army men do battle on a multi-level user designed terrain.  The terrain consisted of 3 elements: walls, platforms, and ramps.  The army men were 2 colors: green and red.  Two players played against each other: each with 12 units.  Units that were within a certain proximity (4") of each other automatically did battle with one another.  

The terrain directly affected the battle results.  Walls between units would occlude line of sight and effectively make units out of range of one another.  The ramps went from table height up to platform height.  Any unit that was more than half a platform height above an opponent would get an advantage (FIXME 2/3 chance of hitting instead of 1/3).  Platforms could only be accessed by walking up a ramp.  

The first step in game play is the placement of terrain.  This is done as a collaborative effort.  Users were free to place the terrain in any way they saw fit.  Following terrain set-up is the soldier placement stage.  Each user places 12 soldiers.  Once again no specific restrictions on soldier setup were placed (although many groups set up guidelines for their own placement).

Each person's turn also had phases.  The first phase is the movement phase.  Each piece could move FIXME 4".  A player can move as many pieces as he wants during his movement phase.  Red always gets the first turn.  Units movements are restricted to distance as possible given the terrain rules e.g. if a unit went up a ramp and had to make a turn to go up a ramp, the distance was measured as the shortest legal move.  After movement, there was a battle phase.  During the battle phase every combination of units battles.  Battles were done with the following probability: a unit is disabled with a 1/3 chance if on the same ground as his opponent and with 2/3 probability if he is on lower ground.

The came continues until all units of one players are disabled at which point the player with units remaining wins.

\paragraph{Augmented and non-augmented versions of play}

We decided that it was important for the game to have a comparison for our user study so we developed a non-augmented version of the game.  In the non-augmented version, not only did they not have the benefit of textures appearing on the primitives but it also affected game play.

In the augmented version dice rolls were simulated.  Users saw a visualization of each battle taking place.  Units disabled had an x placed over them.   Movement circles were projected which showed how far each unit was allowed to move; additionally battle lines were drawn to show units in range of each other.  Please see figure FIXME for pictures of these visualizations.

In the non-augmented version users were given rulers to measure how far units could go as well as to tell if units were in range of each other.  Also dice were rolled for the probabilities of units being disabled.

\subsection{User Study Predictions}


The first goal of our user study was to determine if interaction with
the system was natural and intuitive and to judge the learning curve
for users familiar with physical board games and computers but new to
spatially augmented reality.  A companion goal was to assess the
stability and robustness of our spatially augmented reality system in
a full game scenario by non-developers of the system.  Most
importantly, we wanted to solicit feedback on the visualization
elements and overall game play.  To provide a baseline for comparison,
all study participants played both a traditional, non-augmented
version of the ARmy game as well as the projector augmented game.  We
hypothesized that the augmented version would be less tedious, less
ambiguous or contentious, and that movement and combat would be more
efficient.  Altogether, this would allow users to play more rounds of
the game and explore and evaluate more complex gaming strategies.
% allow users to
%play faster through more rounds of movement and combat, and would
%allow them to experiment with and evaluate different strategies.  
We also hypothesized that users would find the augmented reality
technology more engaging and immersive than the traditional version of
the game.

\vspace{-0.15in}
\subsection
{Goals for User Study Design}

%To have a fair comparison, 
We designed the study as a direct comparison of the same basic game
played two different ways: using traditional non-augmented technology
(rulers \& dice), and using the projector augmentation.  We held
constant the game rules, including the turn sequence, movement
restrictions, combat sight lines, and combat probabilities.
%
After an introduction to the SAR system and a brief description of the
games rules ($\sim$10 minutes), the participants played the game three
times.  The preliminary game was for practice ($\sim$15 minutes)
in which the participants used {\em both} the traditional mechanisms
of rulers and dice and the projected visualizations of movement areas
and combat circles.  In the practice round each player started with 5
army units and we encouraged the participants to set up near their
opponent to ensure they gained experience with the combat rules.  Most
participants played 1 or 2 full cycles of game play (movement for each
player and joint combat after each movement phase).
% to familiarize
%themselves with the rules.
%
Next, the participants played two full games, one with and one without
augmentation, in a randomly-selected order.  For each of the full
games, players started with 12 units each and played for a maximum of
20 minutes.  Participants were specifically {\em not allowed} to use
rulers and dice when playing the augmented game.  Similarly, all
projector visualization and texturing was disabled for the
non-augmented version.
%
The supplementary video shows sample footage of both the augmented and
non-augmented versions of the movement and battle phases of the game.
The script (read aloud to participants) for the user study is included
as supplementary material.

\vspace{-0.15in}
\subsection{Background of Study Participants}

We believe it is important to find participants who enjoy playing
games and have a sense of competitiveness, strategy, and intellectual
curiosity when doing so.
%
Thus, we recruited artists and computer scientists from the Games and
Simulation Arts and Sciences undergraduate major.
%, which consists of
%artists and computer scientists (and dual majors).
%
%
%
%\fbox{can we put this in a table?}
%\fbox{will it make it smaller?}
We had a range of participants from freshmen through graduate
students.  
%
In total, 26 users participated in the initial pilot study (3 females,
5 males) or the main study (6 females, 12 males).  
%We made a few
%revisions to the instructions and questionnaire after the pilot study,
%but the procedure and data collected was similar between the two
%studies.  
We summarize the background for the participants in the main
study: 13 of the 18 participants 
%are studying game design with 
have 0.5-5 years formal education in game studies.  12 of the
participants have at least 3 years formal education in computer
science.  11 of the participants have at least 1 year of formal art
education, 7 have at least 4 years art education.  All users have at
least 3 years experience playing computer games, 14 had more than 10
years experience.  All users have experience playing board games, 15 of
them have more than 10 years experience.  Half of the users had prior
experience (0.5-3 years) with table top games similar to {\em
  Warhammer 40,000}.  
%The varied pool of users provided us with a
%valuable range of feedback from the study.
%\fbox{wording... last sentence necessary?}

\vspace{-0.15in}
\subsection{Written Questionnaire}

\begin{table}[tb]
\begin{center}
\begin{tabular}{@{}l|c@{~}c|c@{~}c|c@{}}
                & \multicolumn{2}{c|}{{\small non-augmented}} & \multicolumn{2}{c|}{{\small augmented}} & {\small $\Delta$} \\
{\small avg. rating    } & {\small all(18)} & {\small 2$^{nd}$only(10)} & {\small all(18)} & {\small 2$^{nd}$only(8)} & {\small all} \\ \hline
{\small acc. distance  } & \small 3.7 & \small 4.0 & \small 4.5 & \small 4.1 & \small 0.8 \\
{\small acc. sight lines} & \small 3.8 & \small 4.0 & \small 4.8 & \small 4.6 & \small 0.9 \\
{\small acc. rules     } & \small 3.8 & \small 3.9 & \small 4.6 & \small 4.4 & \small 0.8 \\
{\small subjectivity   } & \small 4.1 & \small 3.9 & \small 4.6 & \small 4.6 & \small 0.6 \\
{\small interest       } & \small 3.5 & \small 3.9 & \small 4.6 & \small 4.6 & \small 1.1 \\
\end{tabular}
\end{center}%
\vspace{-0.05in}
\caption{Participant's rating of the accuracy of distance
  calculations, line-of-sight judgments, and implementation of rules,
  their assessment of the subjectivity of rule enforcement, and their
  overall interest while playing the game.  Each is scored on a scale
  of 1 to 5, with 5 being the positive attribute quality.
\label{table:questionnaire}
\vspace{-0.1in}
}
\end{table}

At the end of game play, each participant filled out a written
questionnaire (supplementary material)
%.  We asked users to 
directly comparing the augmented and non-augmented versions of the
game 
%by rating on a scale of 1 to 5 
for several important game play characteristics
%of the game
%each of the following characteristics of the
%game:
%\vspace{-0.1in}
%\begin{itemize}
%\item 
%the accuracy of distance calculations,
% \vspace{-0.1in}%\\(1=inaccurate, 5=accurate)\vspace{-0.1in}
%\item 
%the accuracy of line of sight calculations for combat,
% \vspace{-0.1in}%\\(1=inaccurate, 5=accurate)\vspace{-0.1in}
%\item 
%the accuracy of game rule implementation,
% \vspace{-0.1in}%\\(1=frequent errors or confusion, 5=no errors or confusion)\vspace{-0.1in}
%\item 
%the subjectivity of enforcement of the rules, and
% \vspace{-0.1in}%\\(1=subjective/some disagreement, 5=no disagreement)\vspace{-0.1in}
%\item 
%their interest level during game play.
%\\(1=tedious or boring, 5=engaging and fun)\vspace{-0.1in}
%\end{itemize}
%
%The results of these ratings are summarized in
(Table~\ref{table:questionnaire}).  The average rating for all 18 study
participants for each version of the game is provided.  We also
separately average the ratings of the users who played that version of
the game as their \emph{second} playthrough (when they were more
familiar with the game mechanics, rules, and strategy). 
% 8
%participants (4 pairs) played the non-augmented version first.  10
%participants (5 pairs) played the augmented version first.
%
Overall, the ratings indicate that participants were interested in
playing both games, thought that the different game mechanics were
generally accurate, found enforcement of rules was not too subjective,
felt that the game was fair, and rarely disagreed with each other or
with the computer.  
%The participants did consistently rate the
%augmented version of the system more positively on each
%characteristic.  
Using Single Factor ANOVA, users rated the augmented version 
more positively than the non-augmented with a $p$ value of .005 or lower in every 
case except subjectivity.  Users rated subjectivity higher with a $p$ value of .08.

\vspace{-0.15in}
\subsection{Timing Results}


A video camera recorded each experiment
%, which allowed us to
%review the timing data for each game.  We 
allowing us to measure the time for terrain setup, initial army
placement, average time for each player's movement phase (red or
green), average time for a combat round, and average time for a battle
(when two units face off, requiring each player to roll once).  We
also counted the number of rounds (red move/combat/green move/combat)
per game, and the number of battles per game
%\fbox{ Is there a better way to explain what a round is?}
(Table~\ref{table:timing_stats}).  This data allows us to compare the
efficiency of game play with and without augmentation.  We present the
data averaged over all experiments and a separate average of the games
played as the {\em second} full playthrough
%, when players are most
%familiar with the game rules and strategy.  
Note: Due to video errors,
a few of the game recordings are incomplete and 
%are thus 
omitted from these averages.

Game setup is slower with the augmented system for both terrain layout
%(100\% slower) 
and army unit placement,
% (10\% slower).  
due to a number of minor factors: triggering the remote, waiting for
the visualization to refresh, and reminding players to remove
unnecessary materials from the table and step out of the camera's
field of view.  Similarly, movement phases are slower in the augmented
version because moves are validated by the system and the extra time 
required to correct illegal moves.  With SAR system
overhead optimization we believe these differences can be greatly
reduced or eliminated.  In particular, we believe these improvements
combined with user familiarity with the movement region visualization
will allow movement phases to be faster in the augmented version than
in the non-augmented version.

\begin{table}[tb]
\begin{center}
\begin{tabular}{@{}l|cc|cc@{}}
               & \multicolumn{2}{c|}{\small non-augmented} & \multicolumn{2}{c}{\small augmented} \\
\small averages       & \small all(9) & \small 2$^{nd}$only(3) & \small all(9) & \small 2$^{nd}$only(5) \\ \hline
\small terrain setup  & \small 1:16    & \small 1:13         & \small 2:46    & \small 2:07         \\
\small army placement & \small 2:52    & \small 2:38         & \small 3:03    & \small 3:11 \\ 
\small movement phase & \small 0:42    & \small 0:36         & \small 0:52    & \small 0:47 \\
\small combat phase   & \small 1:33    & \small 2:13         & \small 0:22    & \small 0:18 \\
\small single battle  & \small 0:16    & \small 0:21         & \small 0:04    & \small 0:03 \\ \hline
\small \# rounds per game & \small 2.3  & \small 2.0         & \small 3.9     & \small 4.1 \\
\small \# battles per game & \small 26.1   & \small 24.5   & \small  36.6    & \small 41.0 
\end{tabular}
\end{center}%
\vspace{-0.05in}
\caption{ The average timing data (minutes:seconds) for various stages
  of game play are summarized for both non-augmented (traditional) and
  SAR augmented experiments.
\label{table:timing_stats}
}
\vspace{-0.1in}
\end{table}

Not-surprisingly, the augmented system's main efficiency improvement
is gained in the combat simulation phase ($\sim$4X faster).  The
simulation of each battle is visualized one-at-a-time for the players
($\sim$1 second per virtual ``die'' roll).  Note that the combat phase
also includes removal of disabled units.
% at the end of the phase.
The greatest efficiency improvement is in calculating which units have
line of sight and are in range, and {\em most importantly}, in keeping
track of which battles have occurred and correctly accounting for all
combinations of opposing units when they are densely clustered.

As we hypothesized, overall the augmented version allows players to
complete more cycles of game play before time is called (60\% more),
and similarly, more total battles are fought (40\% more) in the
augmented version.

\vspace{-0.15in}
\subsection{Quantitative Analysis of Unit Movements}



Next, we analyzed the army unit movements in augmented games.  We
were interested in quantifying the fraction of unit movements that
were close to the maximum 4'' distance.  We also examined
all cases in which a unit was moved slightly beyond this maximum distance
and marked as an illegal move.  Table~\ref{table:movement_data}
summarizes the data for a total of 678 unit movements.
%
60\% of unit movements were ``big moves'', reaching more than 75\% of
the maximum distance and 25\% of the moves pushed within a pixel of
the 4'' movement radius.  We assume that for most of these moves the
user was strategically interested in maximizing unit movement.
However, they may have shied away from making a full 4'' movement to
avoid having that moved marked as illegal.  9 out of 14 pairings saw
at least one borderline unit movement marked as illegal during either
the practice round or the full game.  When a movement
is marked illegal,
%as a borderline illegal movecheat,
it causes an interruption in game flow, requiring correction of the
error.  In most cases, the player over-corrected for the error,
backing up the soldier to approximately 75\% of the full move.  In a
few cases, the player was confused about which unit moved too far, and
conservatively adjusted several other {\em legal} moves.

Only twice during our game play did the system catch flagrantly
illegal moves, and neither case was malicious.  In one instance the
illegal move came in the confusion immediately after a detection
error.  We found that detection errors during game play were rare,
proving that our prototype system is robust, engaging, and thoroughly
playable.  Detection errors occurred when army units were placed too
close together and detected as a single unit,
% a unit was missed (color threshold),
%red vs. green units were mislabeled 
%mis-recognized red vs. green unit, 
when an army unit hidden behind a wall was not visible to the camera,
and a few rare color thresholding errors.
%person was not visible (behind a
%wall).  
Other detection errors were related to game rule ambiguities; for
example, straddling a unit between platforms that touch only at the
corner or balancing army units on the very edge of a platform.


\begin{table}[tb]
\begin{center}
\begin{tabular}{@{}l@{~~~~~~~~~}r@{~}r|@{~~~~}r@{~~~}r@{~~}c@{}}
                      & \multicolumn{2}{l|@{~~~~}}{\small practice} & \multicolumn{3}{c}{\small full games (14)} \\ \hline
%\small \# games              &     &                         & \small 14   & &     \\
\small total \# moves        & \small 141 &                                & \small  537 & &  \\
%\small avg. moves / player   & \small   5 &                                & \small 19  & &  \\
%moves $<$ 75\%       &  51 & 36\%                    & 222 & 41\% &  (7-70\%)  \\
\small moves $>$ 75\%        &  \small 90 & \small 64\%                    & \small 315 & \small 59\% & \small (30-93\%)  \\
\small moves $>$ 95\%        &  \small 44 & \small 31\%                    & \small 127 & \small 24\% & \small  (0-63\%)  \\
\small borderline illegal    &  \small 8  &                                & \small  4  & &  \\
\small flagrant illegal      &  \small 0  &                                & \small  2  & &  \\
\small detection errors      &  \small 4  &                                & \small  8  & &  \\
\end{tabular}
\end{center}%
\vspace{-0.05in}
\caption{A summary of the individual ARmy unit movement data for the
augmented game play during both practice and full games.
%the
%  augmented version of game play.
\label{table:movement_data}
\vspace{-0.1in}
}
\end{table}




\vspace{-0.15in}
\subsection{Verbal and Written Feedback}

We encouraged participants to 
%to each other and the
%experimenters, 
ask questions and offer feedback throughout the study.  
Each
participant also answered
%completed the written questionnanswered 
several short written answer questions.
% giving us further insight about the positive and
%negative aspects of each game mode.  
%
A summary of the participant comments:
%^Many participants had the same general comments about the games.  A
%^summary of these comments:

%\Vspace.18in}
%\newcommand{\mysep}{-0.08in}

%\begin{small}
{\em Positive Aspects of Traditional Game:} %, Non-Augmented Game:}
%
%\begin{itemize}
%
%\vspace{-0.05in}
%
%\item 
%felt like traditional board game
``It was a little more hands on'' and
``player decisions were more fluid'', which
``keeps players actively involved in game play''.
%\vspace{\mysep}
%
%\item 
Tactile control allowed participants to ``know exact outcome of dice''
and they ``felt responsible for the outcome of the
dice''.
%\vspace{\mysep}
%
%\item 
Users appreciated the
%wiggle room in interpretation of the ryles
``slight bending of rules for more realistic and entertaining game play''.
%it was easier to do certain things, like skip a turn or make mutual agreements on rulings or confusion


%being able to check whether you were in range without waiting for the system to update
%not having to worry about glitches

%\end{itemize}

%\vspace{-0.18in}
{\em Negative Aspects of Traditional Game:} %, Non-Augmented Game:}
%
%\begin{itemize}
%
%\vspace{-0.05in}
%
%\item 
%time consuming
%measuring is tedious
``Much slower combat (actually rolling the dice each time and making sure every combination of soldiers is accounted for)''
%dice kept falling off table 
and
``when many attacks [happened] at the same time, it made the game stop and not be fun''.
%\vspace{\mysep}
%
%\item 
%difficult to keep up with everything during combat
``In a dense soldier cluster there were so many attacks made that we probably lost count at some point and either attacked too many times or too few''.
%``a slightly sloppy feel where i felt that i could be making mistakes''
%sometimes it was tricky remembering which pairs of soldiers had already fired, but we came up with a strategy that mostly solved that.
Participants experienced ``occasional confusion (even about whose turn it was)''
%\vspace{\mysep}
%
%\item 
%some subjectivity with measurements
and ``While we made decisions, they were not necessarily the
correct ones in terms of the rules''.

%\end{itemize}

%\vspace{-0.18in}
{\em Positive Aspects of Augmented Game:} %Game with Projection Augmentation:}
%
%
%\begin{itemize}
%
%\vspace{-0.05in}
%
%\item
Participants found that the ``visualizations were easy to understand'',
it was
%can see where you're going
``easy to see how far you can move'',
%distances were clearly marked as well as line of sight and current shooting pairs.
and it was ``never unclear about whose turn it was or what could be done''.
%``no questioning who was on higher ground''
%\vspace{\mysep}
%
%\item 
%didn't have to do any work
Most participants ``trust [the] computer'' and the games had
``no disagreement between players since there was an ultimate referee''.
%no human errors
%it fixed the "wait did he shoot yet?" problem.  
%\vspace{\mysep}
%
%\item 
%projected scenery made me feel like i was playing a game and not a probability simulator
%visually it was far more interesting
The use of projective texture was  ``more visually stimulating''
and the animations of
``the arrows for combat were amusing to watch''.
%\vspace{\mysep}
%
%\item 
Participants specifically commented that faster, more
efficient play made it ``much easier to establish movement strategy''
%combat was faster
%considerably quicker pace
because ``it allowed you to focus on the game play and not the math
behind it''.

%\end{itemize}

%\vspace{-0.18in}
{\em Negative Aspects of Augmented Game:} %Game with Projection Augmentation:}
%
%
%
%\begin{itemize}
%
%\vspace{-0.05in}
%\item 
``Turns were slow'' while ``waiting for recalculation'' and
  ``moving pieces slightly out of range made it take longer
  sometimes''. 
%\vspace{\mysep}
%
%switching between turns was a little slow.
%
%\item
Occasional system glitches and the ``restriction in terrain placement because of projection'' were negatives.
% (vision)
%minor issues identifying scenery
%some setup bugs
%occasional glitches
%things not getting detected for being in certain positions
%wall blockage issue with camera, couldn't move too close to walls.
``Blind spots forced us to simplify our first terrain design a bit,
but that's a natural consequence of using a single camera''.
%\vspace{\mysep}
%%
%\item
Explanation of game results was sometimes unclear: 
``Don't know the reason the soldier was disabled''.
%\vspace{\mysep}
%
%\item
``Took the player engagement away a bit.  Watched action happen rather
than rolling the dice.  Takes away your feeling of involvement.''
%hard to see line of sight sometimes

%some rules implemented questionable
%\end{itemize}

%\end{small}
%
%\fbox{scalability of color tracking}%
%
%\fbox{* if want more primitives will color tracking work?}
%
%\fbox{Need to better discuss user study data}
%
%\fbox{need to discuss why efficiency is important}


\subsection{Discussion and Conclusions}
%: Suggestions for Designing Spatially Augmented Games}

%Using a single camera and six projectors, 
Using affordable, off-the-shelf hardware we created an immersive SAR
game experience that 15 out of 18 participants preferred over the
non-augmented version.  Two preferred the non-augmented version and
one person called it a tie.
%a clear majority of users preferred to the
%non-augmented version.  
%With the advent of cheaper projectors and
%cameras, 
We believe that many other physical board and card games could
similarly be positively augmented and improved by such a setup.  
%
Depending on the system budget and physical game environment and
components, this infrastructure could be simplified to use a single
projector or expanded to use multiple cameras or alternate tracking
technology (Section~\ref{section:related_work}).  This system is
practical for installation in home ``game room'' environments.
% While our system was able to
%detect our primitives using a single color camera, a system with more
%complex primitives or requiring full 3D reconstruction could benefit
%from multiple cameras or a depth camera.  
%Recent work has even made
%the possibility of using multiple Kinects together to interpret 3D
%scenes~\cite{Maimone12}.  
%Similarly, while our system has 6 projectors
%many applications, such as ones without large occluding walls, could
%be adequately projected by one or two projectors.


Augmented gaming systems should facilitate and encourage users to make
optimal decisions; for example, in ARmy this often means making a full
4'' distance movement.
%Users should also be encouraged to make the most
%effective moves possible.  
Unfortunately, we found that some users were reluctant to make maximal
movements because the delay in correcting an overmove interrupted game
play.
%the system
%often shyed away from
%making the largest moves after being corrected for moving too far.
Instead of forcing users to nudge pieces back, the system could
automatically recognize these slight over-movements, and clamp the
internal representation of the unit's position to the maximum
distance, ensuring complete fairness of game rules application.
%and
%smooth game flow.
%adjust
%the
%Rather than simply
%telling users to correct their moves we plan to modify the system so
%that a piece is automatically moved back to the farthest allowed move.
We recommend that similar tolerances be incorporated when possible in
all AR based games.
%similarly providing some functionality for maximal
%movement in turn-based movement games

Users were excited by the system ease-of-use and efficiency 
of the battle rounds.
%quickly
%battle rounds went.  
%While the battle rounds went very quickly, 
Game setup 
occasionally 
took longer when technical problems occurred
with the prototype SAR system.  These system issues must be resolved
prior to commercialization of this type of technology.
%While users tend to be
%forgiving if they see parts of the system that save time or look
%exciting, care should be taken to always minimize bottlenecks.
It is important that users feel engaged while 
%using SAR systems and
playing SAR games.
%involved with the system.
Even though the combat rounds were more efficient in
%While the time the battles took in 
the augmented version,
% was four
%times faster, 
some users wished they had more control over the automated battle
sequences.
%We have considered limiting how many battles occur between user
%interactions.  
Future studies are needed to determine
the optimal timing of user interactions and automated system computation.
%to see how long is optimal between user
%interactions should be done.  
More visualization of the simulated die rolls and the ability to pause
%interrupt
%We have considered showing the simulated
%die rolls, or allowing movement at some point midway through 
long battle sequences may be beneficial.  Some users noted that the
game state could change significantly between turns.  It was not
uncommon for one half of an army to be eliminated in a single 
%particularly
%intense 
combat round.  More complexity in the game rules, for example ``hit points'' for each unit
would make the game more interesting and require more strategy.
%One potential way to alleviate this would be to have hit
%points or require it to take more than one shot to kill an enemy
%soldier.

%Finally, we suggest having a state diagram that fully encompasses user
%interaction.  This enables fairly accurate predictions of parts of the
%game where there will not be enough user interaction or where there
%may be too much user input before any exciting augmentation.

%\section{Conclusions and Future Work}
%\fbox{I put the negative first}


%\paragraph{Future Work}

%Our SAR system would benefit from 
%several specific improvements, many
%of which point to interesting avenues of future research. 
By adapting our SAR framework to facilitate
%One significant improvement would be to incorporate a method for
simultaneous acquisition and display, the gaming module could more
naturally react
%allowing for the system to react
to users without the need for turn-based update requests.
%A more sophisticated tracking scheme could also lead to improved
%object identification. 
Gesture-based or verbal speech controls, could also be beneficial.
%This part is new
Game play could be made more engaging by improvements to the detection
sequence and in finding ways for users to interact during combat.
%
%Similarly, we believe users would feel greater freedom if we provided a method
%for them to make the maximal moves without requiring adjusting and removing 
%if they went over the movement threshold.
%end newpart
Finally, play could be enhanced by displaying state information
directly onto the units to remove clutter and by adding audio
elements to increase the immersive feel of the game.

Overall, the ARmy application is a fully functional prototype that
demonstrates the key benefits of SAR for tangible gaming.
%and shows how a tangible tabletop
%interface can be combined with a video game module to create an
%entertaining user experience.  
The results of our user study indicate a general participant
preference for the augmented version of our game prototype, and
feedback has shown a positive opinion about SAR technology in the
context of tabletop games.

