%%%%%%%%%%%%%%%%%%%%%%%%%%%%%%%%%%%%%%%%%%%%%%%%%%%%%%%%%%%%%%%%%%% 
%                                                                 %
%                            ABSTRACT                             %
%                                                                 %
%%%%%%%%%%%%%%%%%%%%%%%%%%%%%%%%%%%%%%%%%%%%%%%%%%%%%%%%%%%%%%%%%%% 
 
\specialhead{ABSTRACT}

Tangible User Interfaces combine the exciting field of Computer Graphics with the ability to control computers with physical objects.  TUIs enable more intuitive interaction than is available anywhere else.  Rather than having to become familiar with some new GUI to accomplish computer tasks, TUIs often allow a similar approach to be taken to the natural way of thinking about something; now the computer can interpret the input from the physical objects.  The Virtual Heliodon at RPI is a physical sketching tool designed for architects.  By moving model walls and windows architects can design rooms and view daylighting using a simple system.  This document is a description of my work on this system as well as my work on a photon-mapping renderer for architectural daylighting simulation.
 
\begin{comment}
Tangible User Interfaces are a growing field in Computer Science where the efficiency of using computers to complete everyday tasks is combined with tangibly interacting with the world in order to make an effective interface to do modern tasks.  The RPI Computer Graphics research group has been developing a Virtual Heliodon Tangible User Interface in order to effectively do Daylighting Simulations.  In addition, three user studies have been done to measure the effectiveness of this tool.  This document discusses the system currently in place, surveys the field of Tangible User Interfaces, investigates the field of user studies, and details future work which could be done in this group in this field.
\end{comment}
