\section{User Study Lessons}

Throughout the administering of the three user studies valuable lessons were learned which made each study smoother than the last.  Providing adequate methods for user feedback, treating users consistently, allowing space for mishaps to occur, getting a good user pool and providing a fun experience are all important when administering a study.

In order to treat users consistently, we developed a script to speak from for each study.  While we allowed users to ask questions, we wanted to provide users with as thorough and consistent a description as possible.  Similarly computer scripts started the interaction each time.  The danger of not doing this is that often programs have parameters which can be specified at start-up and it is important that these initial conditions do not bios the user about the system.

User feedback can be gathered in many ways and it is important to use as many as possible so that a full analysis can be done later.  Questionnaires are often the most useful, especially in comparison studies, but if monitoring user interaction is important taking a video can often make observing details possible that can not be monitored while administering a study.  Additionally taking photos at intervals can be very helpful to judge the speed of movements.  Finally all useful comments spoken should be recorded (as video often will not catch all remarks.  The questionnaire should also be thorough, but with as little bios as possible.  Leading questions will not produce useful feedback.  Often open ended questions will be the most useful for finding creative ways to deal with problems in the system or future enhancements.  Ways to quantitatively measure are also important.  Asking users to rate on a numerical scale or to give true/false answers provides a valuable way to discover if there is statistical significance to different user groups.  Finally it is very important that you find an appropriate user pool when running a study.  While random people may help you find the bugs in a program, it will not be useful to see if they can comprehend glare in a daylighting simulation or if they find an augmented version of a tabletop game they would not think to play engaging.  By choosing architecture students and games students for our studies we ensured our users would have adequate knowledge to provide useful insight into our applications.
