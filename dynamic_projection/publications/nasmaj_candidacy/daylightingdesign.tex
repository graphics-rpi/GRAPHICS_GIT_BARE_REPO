\chapter{Daylighting Design Motivation and Goals}

\fbox{This chapter is still in outline form}

Through administering user studies, discussing the system with architects, and evaluating architects use of our daylighting tool we have determined goals for our rendering software \& SAR system.  Ensuring that information is communicated clearly: both in relative dimensions and in terms of brightness, ensuring intuitive interaction, maintaining interactive speed and properly introducing users to a system are all imperative in designing a system.  


Motivations for a daylighting design tool for architectural design  

\begin{itemize}
\item    Reduce carbon footprint of proposed building design
\item    Effective use of stylized daylighting (stained glass windows etc.) can produce beautiful interior illumination.
\item    Daylighting inside a space over the course of the year varies significantly and is difficult to approximate without a daylighting analysis tool.
\item    Daylighting analysis is currently done in architectural design, but often too late for improvements to be made.  This interface is designed to be used in the sketching phase of architectural design.
\end{itemize}

System Requirements     

\begin{itemize}
\item    Produce daylighting measurement approximations at interactive rates.  The renderer must be a predefined interface to take parameters and return a rendering to our collaborators software.  Our collaborators need the ability to measure the amount of light on a given area across many times/days over the course of the year.  Our renderer needs to be able to produce these values in reasonable time (15-20 seconds for 56 times during the year).  We also need to provide a suitable interface to provide renderings. The renderer must have at least a diffuse material model and should be accurate.
\end{itemize}


