%%%%%%%%%%%%%%%%%%%%%%%%%%%%%%%%%%%%%%%%%%%%%%%%%%%%%%%%%%%%%%%%%%% 
%                                                                 %
%                            ABSTRACT                             %
%                                                                 %
%%%%%%%%%%%%%%%%%%%%%%%%%%%%%%%%%%%%%%%%%%%%%%%%%%%%%%%%%%%%%%%%%%% 
 
\specialhead{ABSTRACT}
 
% \fbox{Will write abstract}

This thesis presents a detailed look at terrain modeling through the lens of mathematical analysis, and explores representation of the terrain surface 
through mathematical operations, descriptive statistics, and erosion simulation.
Modeling terrain surfaces is an essential aspect in a variety of applications in Geographic Information Science, from path planning to dam siting to terrain compression and beyond.
This work presents a novel representation of terrain that maintains its shape properties while facilitating the representation of hydrographically valid terrains, capturing the richness of the physics of the terrain's generation by digging channels in the surface.
% The accuracy of this representation is tested with a series of terrain distance metrics, including an adaptation of the Hausdorff distance.
Terrains are also described by a series of statistical characteristics known as the terrain fingerprint. These characteristics describe the behavior of water flow on the surface, and in turn can be used to compare terrain datasets.
This work also presents a simulation of hydraulic erosion on earthen embankments, motivated by the catastrophe in New Orleans during Hurricane Katrina. The generation of terrain data through erosion simulation is a well-studied problem, but the goal of the simulation in this work is the accuracy of the formation, and as such it is validated with laboratory experiments.
Finally, this thesis presents an application for exploring and manipulating terrain data in a group collaboration environment through the use of a novel multi-user laser tracking and identification system.
% This work explores the mathematics of terrain data through representation, generation, and statistical description.

