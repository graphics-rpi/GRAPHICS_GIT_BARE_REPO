\documentclass[12pt]{article}

\usepackage{epsfig,url,palatino,color}

\pagestyle{plain}

\renewcommand{\baselinestretch}{1.}

\setlength{\topmargin}{0in}
\setlength{\evensidemargin}{0in}
\setlength{\oddsidemargin}{0in}
\setlength{\headheight}{-0.25in}
\setlength{\headsep}{0in}
\setlength{\footskip}{0.2in}
\setlength{\textheight}{9in}
\setlength{\textwidth}{6.5in}
\setlength{\parindent}{0in}
\setlength{\parskip}{0.15in}

\renewcommand{\topfraction}{0.99}
\renewcommand{\bottomfraction}{0.99}
\renewcommand{\textfraction}{0.01}
\renewcommand{\floatpagefraction}{0.01}
\renewcommand{\dbltopfraction}{0.99}
\renewcommand{\dblfloatpagefraction}{0.01}

\pagestyle{empty}

\begin{document}

{\bf SUPPLEMENTARY VIDEO GUIDE  }

We present our paper, “Evaluation of a Tangible Interface for Architectural Daylighting Analysis”.  

{\em Modeling}

Our tool is designed to allow modeling to be a quick sketching process with simple wall and window primitives.  You can see how in this model the user used both exterior and interior walls as well as a single window and an arrow to indicate the north direction.  

Note that bright room lights are switched on \& off for clarity during video filming.  During normal system use a dimmer ambient room lighting is sufficient for model construction and editing and this ambient light is dimmer than the projected visuals and is also left on during projection of simulation results.

{\em December 21}

Simulations can be requested either for a specific time or for a time lapse of  a day of the year.  This video shows a time lapse for the created room for the latitude and longitude where the user study took place.

{\em March 21 }

You will notice, the corner on the left of the video is particularly dim.  Also notice that in the corner near the window, glare is a problem in the morning and as the day goes on the area of glare shifts across the room.  In winter months the glare reaches all the way to the interior wall of the room during the evening hours.

{\em Renovation}

Our tool allows users to quickly see how renovations will affect existing spaces.  In this video you can quickly see the effects of adding additional windows to the example lab space.  Many users attempted a similar renovation.  Unfortunately this renovation presents a significant  glare problem to the space.  We intend to add false color renderings to our system in the future so that users will be more aware of glare problems.

\newpage

{\bf DAYLIGHTING IN AN OPEN PLAN OFFICE SETTING }

\vspace{0.3in}

{\bf Existing program:}\vspace{0.1in}\\
\hspace*{0.2in}dimensions: ~~approx. 25 feet x 32 feet,  800 sq feet. \\
\hspace*{0.2in}ceiling height: ~~approx. 11 feet tall.\\
\hspace*{0.2in}single window: ~~faces due south, 4 feet wide x 8 feet tall.\\
\hspace*{0.2in}occupancy: ~~desk space for 12 students.\\
\hspace*{0.2in}working hours: ~~approx. 10am - 6pm,  year-round.

\vspace{0.4in}

{\bf Your tasks for today's study:}
\vspace{-0.1in}

\begin{enumerate}
\item {\bf Analyze the available daylighting in the current room
  design.}  Identify areas in the room that have too much or too
  little daylighting.  Identify areas and times when glare from the
  sun might be problematic for students sitting at the desks trying to
  do computer work.

  \begin{enumerate}

  \item First, using your intuition alone, consider how daylighting
    will affect this space.

  \item Next, build a scale model of this room in its current form
    using the virtual heliodon and analyze the available daylighting,
    testing your hypotheses.
    
  \end{enumerate}

\item {\bf Suggest renovations to the room to improve the use of
  daylighting.}  Make these edits to the design using the virtual
  heliodon and re-analyze the available daylighting in the new
  space.

\item {\bf Create and analyze a completely new design for the same
  program. } Your new design should provide working space for 12
  students with roughly the same square footage, but it can be located
  in a different building on campus, have a different orientation with
  respect to the sun, etc.  Use the virtual heliodon to sketch and
  analyze your new design.

\end{enumerate}

After each exercise you will be asked to complete a few short written
questions.


\newpage 
{\bf PART 1: INTUITION OF EXISTING DESIGN }
\hfill Participant ID \verb+_______+
\vspace{0.2in}

Identify {\em areas of the room, times of the day, and days of the
  year} that will have: (A) too much illumination from daylighting,
(B) too little illumination from daylighting, and (C) the potential
for glare from the sun.  Make a quick sketch of the existing room and
annotate this sketch with your predictions.

\vspace{4.8in}


Based on your intuition, estimate the percentage of normal working
hours throughout the year that the desks receive sufficient
illumination from the sun and sky alone (no electric lighting) to
perform typical office work.  (Daylight Factor/Daylight Autonomy)


\vspace{0.9in}


Describe the available daylighting and use of electric lighting within
the room during your visit.  How did the current condition affect your
impression of the space?

\vspace{0.6in}


\newpage
{\bf PART 2: ANALYSIS OF EXISTING DESIGN }
\hfill Participant ID \verb+_______+\\
{\bf WITH VIRTUAL HELIODON }
\vspace{0.2in}

What time and day simulations or time-lapse animations did you request
for analysis?  What was your strategy in selecting these moments or periods?
\vspace{1.7in}


Did you understand the resulting simulation display?  What was
confusing or unclear in the simulation?
\vspace{1.65in}


Based on your analysis with the new tool, what is your new estimate of
the Daylight Factor/Daylight Autonomy?  Explain any difference in your
previous estimate.
\vspace{1.65in}



What new insights did you gain about daylighting within this space?
Were any of the simulation results unexpected?  
\vspace{1.7in}





\newpage

{\bf PART 3: ANALYSIS OF A PROPOSED RENOVATION } \hfill Participant ID
\verb+_______+
\vspace{0.2in}

Describe your proposed renovation.  What was your strategy to improve
the use of daylighting within the space?
\vspace{1.7in}


What time and day simulations or time-lapse animations did you request
for analysis?  Did the proposed renovation perform as expected?
\vspace{1.7in}



What is your estimate of the Daylight Factor/Daylight Autonomy of the
new space?
\vspace{1.7in}


On a scale from 1 (poor) to 5 (excellent), rate the effectiveness of
this new tool for:

1~~~2~~~3~~~4~~~5~~~ Evaluating the quantity of illumination (too much or too little)

1~~~2~~~3~~~4~~~5~~~ Determining the potential for glare at different locations and time periods

1~~~2~~~3~~~4~~~5~~~ Understanding the interesting and dynamic qualities of daylighting

1~~~2~~~3~~~4~~~5~~~ Use in architectural education for daylighting study and analysis



\newpage

{\bf PART 4: ANALYSIS OF A NEW DESIGN }
\hfill Participant ID \verb+_______+
\vspace{0.2in}


Describe your new design and the motivations behind this design.  
\vspace{1.7in}


Were you able to build a satisfactory model of the design?  If not,
what aspects of your new design that are important for daylighting
simulation and analysis were you unable to model?  
\vspace{1.7in}



What is your estimate of the Daylight Factor/Daylight Autonomy of the
new space?  Overall how did the new design perform and how satisfied
are you with these results?
\vspace{1.7in}



Additional Feedback: Please describe your suggestions for how we can
improve the effectiveness of this tool for daylighting analysis.
\vspace{1.7in}





\renewcommand\arraystretch{1.0}


{\bf SCRIPT FOR DAYLIGHTING IN OPEN PLAN OFFICE SETTING}
\vspace{0.1in}

Thank you for volunteering to participate in today's user study.  We
are studying the effectiveness of our new physical daylighting
simulation tool for architectural design and analysis.  We call this
tool the virtual heliodon.

Today you will be asked to perform a few design exercises using our
virtual heliodon setup.  

Let's go across the hall and I'll show you the space which you will be
asked to first analyze in its current form, and then redesign.

{\em WALK FROM TUI INTO EXAMPLE ROOM }

For today's user study you will be asked to complete a few short
daylighting analysis and design exercises related to this student
office space using the virtual heliodon.  Here is an overview of the
exercises.  After each exercise you will answer a few short written
questions.

{\em HAND USER EXERCISE SHEET }

This office area is for computer science graduate students studying
computer graphics and computer vision.  As you can see, this room is
arranged for about a dozen students working at laptop or desktop
computers with standard LCD monitors.  Typical working hours for
students in the lab are from 10am-6pm, but it tends to be more busy in
the afternoons.  As you can see the room contains a single window,
which faces almost due south.  The dimensions of the room are roughly
25 feet by 32 feet.  The ceiling is roughly 11 feet above the floor
and the window is 4 feet wide by 8 feet tall.

\newpage 

Briefly, your tasks for today's study are as follows. 

\begin{enumerate}
\item {\bf First, you will analyze the available daylighting in the
  current room design.}  You will be asked to identify areas in the
  room that have too much or too little daylighting, and to identify
  areas and times when glare from the sun might be problematic for
  students sitting at the desks trying to do computer work.

   You will do this analysis first from your intuition alone, and
   second by using the virtual heliodon.

\item {\bf Next, you will suggest renovations to the room to improve
  the use of daylighting.}  You will make these edits to the design
  using the virtual heliodon and re-analyze the available daylighting
  in the new space.

\item {\bf Third, you will create and analyze a completely new design
  for the same program. } Your new design should provide working space
  for 12 students with roughly the same square footage, but it can be
  located in a different building on campus, have a different
  orientation with respect to the sun, etc.  You will again use the
  virtual heliodon to sketch and analyze your new design.

\end{enumerate}

Now I will give you a chance to explore this room, make notes about
the existing room geometry and materials, and ask questions.  Do you
have any questions about the existing space or how we are using the
room?

{\em PAUSE TO LET THE USER EXPLORE THE ROOM AND ANSWER ANY QUESTIONS }

Please fill out Part 1 of the questionnaire.

{\em PAUSE FOR USER TO FILL OUT QUESTIONNAIRE  --  APPROX. 5 MINUTES }

Ok, now we will return to the other room.

{\em RETURN TO TUI }

Next, we would like you to build a model of the existing room
geometry using the virtual heliodon system.  

There are three types of geometry elements you will be using today:
flat walls, curved walls, and window markers.  The approximate scale
of these components is 1 inch = 1 foot.  The cyan markers are for tall
windows, the yellow markers are for short windows.  

To construct a design, you may place elements on the table
surface. Please make sure to keep the elements within the pencil
circle, so that the overhead camera can distinguish each element.
Note that walls need not touch to indicate a connection.  These gaps
will be automatically detected and the system will build a closed
model of the geometry for simulation.

There are 2 types of tiles you will place on the table surface.  The
first is the ``north arrow'', which indicates the overall orientation
of the building on the site.  The second type of tile indicates the
materials within the design.  The small paint chip in the center of
the tile indicates the color of the material.  A tile with a green
border indicates the floor material.  A tile with a blue border
indicates the wall material.  Please place these tokens sufficiently
far away from the walls of the design so that the overhead camera has
a clear view of each token.

Now let's get started.  Please create a scale model of the geometry
and materials of the office space we are analyzing today.

{\em PAUSE FOR USER TO BUILD THE DESIGN.  PROMPT THE USER TO ADD THE
  WINDOW, NORTH ARROW, AND MATERIALS IF THEY HAVE OMITTED ANY
  COMPONENT. }

Now we are ready to do analysis.  You can request from me a simulation
of the daylighting for any time of day for any day of the year for a
clear sunny sky.  Please tell me the month, day, and time of day you
would like to see.

{\em PRESENT THE REQUESTED SIMULATION TIME/DAY }

You may also request a time-lapse animation for a single day.  Please
tell me the month, day, and the start and end times for the animation.

{\em PRESENT THE REQUESTED TIME-LAPSE ANIMATION }

Please continue with your analysis of the daylighting within this
design.  Just let me know what times and dates you would like me to show
you.

{\em PERFORM THE REQUESTED SIMULATIONS.  PROMPT THE USER TO PERFORM
  SEVERAL SIMULATIONS.  -- APPROX 5-10 MINUTES TOTAL ANALYSIS TIME }

Please fill out Part 2 of the questionnaire.

{\em PAUSE FOR USER TO FILL OUT QUESTIONNAIRE  --  APPROX. 5 MINUTES }

Next, we would like you to suggest some renovations to the existing
space that will improve the use of daylighting within the design.
Make the appropriate modifications to the geometry and materials of
the design.  Let me know when you are finished.

{\em PAUSE FOR USER TO BUILD A NEW DESIGN.  IF THE USER TRIES TO
  CHANGE THE NORTH ARROW, OR MAKE MAJOR CHANGES TO THE DESIGN, REMIND
  THEM THAT THIS SHOULD BE A RENOVATION OF AN EXISTING SPACE.  }

Now we are ready to do the daylighting analysis of the design.  Again,
please request any single time/day simulations you would like to view
or time-lapse animations you would like to view.

{\em PERFORM THE REQUESTED SIMULATIONS.  PROMPT THE USER TO PERFORM
  SEVERAL SIMULATIONS.  -- APPROX 5-10 MINUTES TOTAL ANALYSIS TIME }

Please fill out Part 3 of the questionnaire.

{\em PAUSE FOR USER TO FILL OUT QUESTIONNAIRE  --  APPROX. 5 MINUTES }

Finally we would like you to be creative and design a brand new
working space that will better serve the needs of the graduate
students with respect to daylighting.  You may site this space
anywhere on campus, you are not restricted to renovating the existing
space.  Let me know when you are finished.

{\em PAUSE FOR USER TO BUILD A NEW DESIGN. }

Now we are ready to do the daylighting analysis of the design.  Again,
please request any single time/day simulations you would like to view
or time-lapse animations you would like to view.

{\em PERFORM THE REQUESTED SIMULATIONS.  PROMPT THE USER TO PERFORM
  SEVERAL SIMULATIONS.  -- APPROX 5-10 MINUTES TOTAL ANALYSIS TIME }

Please fill out Part 4 of the questionnaire, and finally your name,
address and consent for publication.

{\em PAUSE FOR USER TO FILL OUT PAPERWORK  --  APPROX. 5 MINUTES }

Thanks for your participation in the study.  I'd be happy to answer
any questions you have about our research.  


\end{document}
