\documentclass[10pt]{article}


%\usepackage{fullpage,epsfig,wrapfig,url,palatino,color}
\usepackage{epsfig,wrapfig,url,palatino,color}

\pagestyle{plain}

\renewcommand{\baselinestretch}{1.}

\setlength{\topmargin}{0in}
\setlength{\evensidemargin}{0in}
\setlength{\oddsidemargin}{0in}
\setlength{\headheight}{0in}
\setlength{\headsep}{0in}
\setlength{\footskip}{0.2in}
\setlength{\textheight}{9in}
\setlength{\textwidth}{6.5in}

\renewcommand{\topfraction}{0.99}
\renewcommand{\bottomfraction}{0.99}
\renewcommand{\textfraction}{0.01}
\renewcommand{\floatpagefraction}{0.01}
\renewcommand{\dbltopfraction}{0.99}
\renewcommand{\dblfloatpagefraction}{0.01}

\begin{document}

\noindent
{\bf Title of Proposal:}  Evaluation of a Multi-User Collaborative Audio Workstation and Expirementation Environment\\
{\bf Researcher:}   Tyler Sammann\\
{\bf Address:}  MRC 309A\\
{\bf Phone:} 860 918 5972\\
{\bf Research Advisor (for students):}  Barbara Cutler \\
{\bf Department:}  Computer Science \\
{\bf Is this proposal related to a sponsored project?}  Yes \\
{\bf If yes,  please indicate:}  \\
Existing Award: (Fund \# A12016), NSF, \\
Immersive Architectural Daylighting Design Experience \\

\noindent
All investigators, including faculty supervisors, on this project must
complete the self-study course on protection of human research
subjects. \\
{\bf Certification:  I/We have completed the course:} \\
Tyler Sammann (CS Masters student) 8/22/12 \\
Barbara M Cutler 7/2/08, refresher 11/2/11

\paragraph{Objective:}
%
To evaluate the effectiveness of our table top Spatially Augmented
Reality (SAR) system for multi-user interaction and data
visualization.  Specifically we will study the effectiveness of the
visualization and interaction design for a simple 2-player board game
with projector augmentation and computer-based evaluation of the game
rules and game play.


\paragraph{Methods:}
%
The participants will be asked to play two to four rounds of a simple
2-player board game with small cardboard and plastic pieces on a table
augmented with visualization by projectors and answer a short
post-play questionnaire.  Both players will simultaneously be
participants in the study.  We will use a video camera with audio to
record the game play and save digital files of the game state during
play.  During the exercise, participants will be asked to speak aloud
to each other and to the researcher about their observations of the
system and the overall interaction.  After the game play is completed,
each participant will be asked to fill out a paper and pencil
questionnaire about the SAR system, the game interaction design and
visualization, and its usefulness in providing feedback and
implementation of the game rules and game play.  The system
introduction will take approximately 10 minutes, game play will last
approximately 30 minutes (5-10 minutes per round), and the written
questionnaire will take about 20 minutes to complete.  The entire
study (system and game introduction, game play, and questionnaire)
will last approximately 1 hour.  We will ensure each participant is
completed with the entire study in a maximum of 1.5 hours.

Our primary data collection device is the post-study paper
questionnaire.  If the participants allow, we will also video tape the
session.  The video tape is not necessary for completion of our study.
Participants that allow us to video tape may be used as samples in our
paper, and will allow us to do further analysis related to the timing
of game actions, etc.  The data collected from video taping and audio
taping will be fully anonymized before publication.  Faces and bodies
will not be visible in the imagery.  And we will not publish the audio
files, only possible extracted quotations.

The participant will be told that they are under no obligation to
participate in the study, and that they may withdraw from the study at
any point, without giving a reason.  

Participants for this study will be compensated for their time in the
form of a gift certificate at the rate of \$10 per hour.  This
compensation is not contingent upon the subject completing the entire
study and will be prorated if the subject withdraws.  The
participant's performance in the game (winning, losing, playing well
or poorly) will not affect compensation.

\paragraph{Effects on Subjects:}
%
See benefits and risks.

\paragraph{Benefits to Participants:}
The participants will gain first-hand experience with a new
technology, Spatially Augmented Reality (SAR), and an application of
SAR to visualization and multiple player interaction for game play.
The results of the study will lead to the use, development, and
improvement of spatially augmented reality tools for entertainment and
education.

\paragraph{Documentation of Risks:}   
The spatially augmented reality system is identical to that from our
earlier user study \#894 “Evaluation of the Virtual Heliodon for
Architectural Daylighting Design.  The risks are the same for these
two studies.  We will follow the same safety precautions and
participant instructions as in that study.

The participants will be standing around a small table, manipulating
small cardboard objects on the table and observing imagery projections
on the table surface.  The table is surrounded by a heavy duty
aluminum truss frame with 6 projectors mounted on the frame.

There is a risk of permanent eye damage if participants stand close
to (30 centimeters or less) and within the beam of the projectors and
look directly into the lens for more than 2 seconds.  The study does
not require participants to position themselves or direct their gaze
in such a way that this damage would occur.

\paragraph{Measures to Minimize Risk:}
\vspace{0.1in}
\noindent
The study will be conducted in a quiet mixed-use lab/office space in
the Materials Research Center.

See attached ``Overview of the Table Top Spatially Augmented Reality
System'' for a description of the physical system.  As a physical
environment, the system does pose some minor physical risks, but we
have taken all steps to minimize these risks, described below.  One or
more researchers will be present at all times and will stop the study
immediately if the equipment or safety mechanisms are not fully
functional or if the participant is experiencing any difficulties.

\begin{itemize}

\item All joints, bolts, nuts, and screws of the frame and peripherals
  (projectors \& camera) will be checked for tightness prior to the
  user study.  Before each participant begins the user study, the
  investigators will ensure that the frame and equipment is secure and
  stable, and that the area within and around the frame is clear.

\item The participants will be given an overview of the system and all
  components of the system will be described to them.  The will have
  an opportunity to ask questions about the system before the study
  begins, and the participants may ask questions during the study as well.

\item The participants will be instructed not to look directly into
  the projectors.  The nature of the setup and the exercises the
  participant is asked to do makes it very unlikely that the user
  would accidentally look directly at the projector bulb.  The natural
  gaze while using the table-top SAR system is directed downwards,
  towards the table, and all projectors are mounted above head height.
  Furthermore, the placement of the table in the center of the space
  at the convergence of the beams of the light, effectively blocks a
  user from positioning him/her self within the beam.  Before the
  experiment begins, we will caution the participants against looking
  directly at the bulbs and explain how to stand and move within the
  frame to avoid accidentally doing so.  During the experiment if the
  participant appears to be at risk for looking directly into the bulb
  we will stop the experiment.

\item We will provide and offer simple opaque visors to the
  participants, similar to a baseball cap, for use during the study.
  The visor will further protect the participant from accidental
  exposure to the projector light from above.  However, we will not
  require that the participant wear the visor during the study.
  Whether the participant is wearing the visor or not, we will be
  watching the participant carefully and stop the experiment if the
  participant is not using the system as intended and is in any risk
  of positioning themselves for risk of eye damage.

  We note that during intended use of the system, the user's head
  should not be in or near the beam of light from any projector.  The
  user naturally positions himself between the beams of the
  projectors.  During normal use of the system, if the user does move
  to step into the beam, the beam will fall on the person's hand or
  arm, or possibly their back.  Standing close to the table will
  ensure that the user's head is never in the beam.  If the user does
  step into the beam, this is immediately apparent because it casts
  shadows on the table (the user's gaze is towards the table) and the
  user naturally moves out of the beam again.  If the user ever moves
  towards placing their head within the beam, we will stop the
  experiment.


\end{itemize}

\noindent 
The entire study will last approximately 1 hour.  The participants
will be encouraged to work at their own pace and take breaks as needed
to stretch or sit down (chairs will be available in the room) and will
be told they may stop the study at any time without giving a reason.
The study will consist of 2-4 rounds of game play.  Each game consists
of a period of roughly 5-10 minutes standing at the table-top SAR
system followed by a break where the user may sit at a desk in the
room to answer written or verbal questions and receive instructions
for the next exercise.  If a participant has not completed all game
exercises within 1 hour, we will stop the game portion of the session
and have the user complete the post-game questionnaire.  Similarly, we
will ensure that the participant completes the entire process within
1.5 hours.

\paragraph{Likelihood of Harm:}   Very minimal.

\paragraph{Alternate Method Not Using Human Subjects:}
None

\paragraph{Qualifications of Researcher:}
Barbara Cutler has a PhD in Computer Science from Massachusetts
Institute of Technology.  Joshua Nasman is a 4th year PhD student in
Computer Science at Rensselaer Polytechnic Institute studying computer
graphics and parallel computing.

\paragraph{Recruiting of Subjects:}
We will ask for student volunteers, 18 years or older, from the Games
and Simulation Arts and Sciences courses and major in the School of
Humanities, Arts, and Social Sciences.  We will obtain permission of
the course instructors to advertise for the participation of their
students, but participation in the study will be voluntary and will
not impact their course grade.  The faculty advisor for the study
(Barbara Cutler) will not recruit students in her courses to
participate.  The names of the students who did or did not participate
in the study will be confidential and will not be released to their
instructors.

We are specifically interested in studying people with interest in the
field of Games and Simulations Arts and Sciences (GSAS) and will
advertise the study to GSAS majors.  Most or all students in this
major have extensive prior experience using computer software,
computer games, and graphical user interfaces.  Most or all students
in this major are also very familiar with board games and many are
familiar with specific ``miniature war games'', similar to the game
used in our study.  This experience is not required for participation
in the user study, but the subject's interest and prior experience
with these interfaces and games is noted, and thus makes participation
in this study simple and fun for the participant.  


\paragraph{Confidentiality:}
Participants will be identified by a randomly assigned ID number that
is used only for this study.  All recordings and design files will be
labeled with this ID (and not the participant's name).  All
information and data relating to the user study will be protected to
secure confidentiality.  All electronic files will be stored on
password protected computers in locked offices, which can be accessed
only by the investigators of the user study.  All paper forms (e.g.,
the exit questionnaire) will similarly be labeled with the ID and not
name.  The paper forms will be stored in Barbara Cutler's locked
office.  The correspondence between ID number and participant name
will be recorded by Barbara Cutler and stored on a password protected
computer, accessible only by the her.  This correspondence will be
destroyed once analysis of the data is complete, within 1 year after
participation in the study.  The video/audio recording will be
destroyed within 1 year after participation in the study.

\newpage

\section{Overview of the Table Top Spatially Augmented Reality System}

%\noindent
%The proposed user study investigates the effectiveness of the
%visualization and multi-player interaction of the ``ARmy'' game
%developed by Andrew Dolce~\cite{dolce} for his Computer Science
%Masters thesis (Figure~\ref{FIGURE_army_screenshots}). The ARmy
%application is a military simulation game played between two
%opponents. The game is in many ways similar to a typical tabletop war
%board game, in that it uses miniature objects to create a physical
%representation of the game world. The players are given a number of
%plastic soldier figurines or units, which represent their respective
%armies. Each player moves his units through the scene according to the
%rules of the game, engaging in combat with opposing units to eliminate
%them from play.

%\begin{figure}[h]
%\centering
%\resizebox{5in}{!}{\includegraphics{army_screenshots}} \\
%\caption{ Imagery from Andrew Dolce's ARmy military simulation game
%  played on our Table Top Spatially Augmented Reality System.}
%\label{FIGURE_army_screenshots}
%\end{figure}

The participants position a set of small boxes, ramps, and partition
walls made of lightweight white foamcore (foam+cardboard) and red or
green plastic army units within the workspace to construct the 3D
geometry of the game environment and control game play.  Images of the
environment are captured by a camera mounted above the scene and are
processed using computer vision to detect the current game state and
create a 3D digital model.  The ARmy computer application calculates
the legal movement range of each unit and also calculates the line of
sight and distance between opposing units to determine if combat will
occur between opposing units and if an advantage is afforded to one
unit over the other because of elevation differences.  The projectors are used to
directly augment the physical objects:
\begin{itemize}
%\item adding realism to the environment by applying interesting textures
%to the surfaces (Figure~\ref{FIGURE_realism}),
%\item
% visualizing the legal movement range (Figure~\ref{FIGURE_movement}), and
%
%\item visualizing combat line of sight between opposing army units
%(Figure~\ref{FIGURE_combat}).
%\end{itemize}
%
 The computer also enforces the rules of the game
(play is paused until illegal movements are corrected).

%\begin{figure}[t]
%\centering
%\resizebox{2.5in}{!}{\includegraphics{complex_with_army_lights_on}}
%\resizebox{2.5in}{!}{\includegraphics{complex_projection_with_army}} \\
%\caption{ The projectors are used to add realism to the plain white
%  surfaces by projecting simple building textures onto the physical scene.}
%\label{FIGURE_realism}
%\end{figure}

%\begin{figure}[t]
%\centering
%\resizebox{2.5in}{!}{\includegraphics{movement_terrain_projection_lights_on_with_army}}
%\resizebox{2.5in}{!}{\includegraphics{movement_circles_projection_lights_on}} \\
%\caption{ The legal movement for each army unit is computed and visualized
%  directly on the physical scene using projection imagery.}
%\label{FIGURE_movement}
%\end{figure}

%\begin{figure}[t]
%\centering
%\resizebox{2.5in}{!}{\includegraphics{combat_movement_u}}
%\resizebox{2.5in}{!}{\includegraphics{combat_movement_ze}} \\
%\caption{ The line of sight between opposing army units is computed
%  and visualized by projecting yellow lines connecting army units that
%  are sufficiently close and do not have an obstacle between them. }
%\label{FIGURE_combat}
%\end{figure}







%\begin{figure}[t]
%\centering
%\resizebox{!}{3.2in}{\includegraphics{contraption_frame}}
%\resizebox{!}{3.2in}{\includegraphics{contraption_with_people}} \\
%\caption{ The Table Top Spatially Augmented Reality System consists
%  of 6 projectors and 1 camera mounted on a frame of heavy-duty
%  aluminum truss frame commonly used in stage technology settings.
%  Users gather around a table in the center of the frame to play the
%  game and view visualization imagery projected on the 3D environment
%  built of white foamcore.}
%\label{FIGURE_contraption}
%\end{figure}



%Our system (Figure~\ref{FIGURE_contraption}) centers around a standard
%30'' high, 42'' diameter table.  A standard, heavy-duty stage
%technology aluminum truss frame positions the camera and projectors
%above the table.  The projectors and camera are securely mounted to
%the frame using appropriate, commercially-available mounting brackets.
%The wide stance of the frame allows more than 26'' of clearance
%between each of 4 vertical uprights and the table.  Three of the four
%sides of the frame are open with at least 3' of clearance.  The top
%bar of the frame is 7' above the base.  

%Six standard portable office projectors (each weighing approximately
%7.5 lbs) are mounted in a circle above the heads of the users using a
%standard mounting bracket rated to hold projectors of this size.  The distance
%from the floor to the bottom of the projector is over 6'6''.  A
%digital camera, weighing approximately 0.5 lbs, is positioned
%approximately 8' above the floor and is secured with a standard tripod
%mounting bracket.  Standard power and data cables are used to connect
%the projectors, camera, and lights to three computers placed on a desk
%near the frame.  The cables are neatly run along the upper bars of the
%frame and attached using cable ties and the excess cable is looped and
%secured next to the desk.  No cables are loose or are run along the
%floor.  The system contains no custom electrical components and all
%electrical components are used within their UL design specifications
%and according to the manufacturer's instructions.

%Standard fluorescent tube room lighting is used during operation of
%the table top Spatially Augmented Reality system and will be used
%throughout the duration of the study.  The average illumination in the
%space is 300 lux (lumen/m$^2$).  For reference, recommended lighting
%levels for normal office and laboratory work is in the range of
%250-1000 lux or for precision and detailed work the recommended range
%is 1500-2000 lux.  With room lighting only, the illumination on the
%table surface is 300 lux.  With all 6 projectors displaying full
%brightness white images, the center of the table surface is less than
%20,000 lux.  For reference, noon sunlight falling on a horizontal
%surface is approximately 120,000 lux.  All objects that will be placed
%on the table are matte (not reflective or shiny) and thus the
%brightness from the projectors and halogen lights will be uniformly
%distributed in all directions and will not result in any intensely
%bright specular reflections to any viewpoint.

%\begin{figure}[t]
%  \begin{center}
%    \includegraphics[width=0.45\textwidth]{top_view.pdf} \hfill
%    \includegraphics[width=0.5\textwidth]{side_view.pdf}
%  \end{center}
%  \caption{Diagram of system of system from the top (left) and side
%    (right).  The natural place for users to stand during use of the
%    system is next to the table between two uprights rather than
%    between the table and one upright as this would put them in the
%    beam of light and cause shadows on the table. The natural gaze of
%    participants is downwards towards the table or horizontally to
%    make eye contact with other users rather than directly into the
%    projector lens.
%\label{FIGURE_system_diagram}
%}
%\end{figure}

%\section{Safety of Projectors}

%We are using 6 Optoma EP 727 projectors.  Each projector uses a 180W
%Phillips UHP mercury lamp.  These projectors are typical small
%portable projectors that are used for everyday presentations.  We have
%not made any physical modifications to the case of the projector, thus
%all manufacturer safety features to protect users in the event of a
%bulb failure or breakage are still in place.  Similarly, we have not
%made any modifications to the intensity of light output from the
%projector or removed or changed any of the filters and lenses
%installed in front of the bulb.  Thus, intuitively, because it is safe
%to observe (for extended periods of time) the screen upon which the
%image is projected, logically it is also safe to observe the table and
%wall surfaces upon which these projectors display images.  We further
%analyze this safety in the following section.

%It is not recommended to stand in the beam of projection and look
%directly at the projector.  Use of our system and our specific user
%studies will never require or ask the users to gaze toward the
%projector lens (Figure \ref{FIGURE_system_diagram}) and we will
%caution them both verbally and through signage about the potential
%eye hazard in doing so (Figure~\ref{FIGURE_warning_signs}).  However,
%we note that during normal use of these and similar projectors for
%class or office presentations, the presenter often does just this as
%he faces the audience.  Thus, our system is well within the normal,
%expected, and safe use of these devices.


%\begin{figure}[t]
%  \begin{center}
%    \includegraphics[width=0.3379\textwidth]{img_1193_small2.jpg}
%    \includegraphics[width=0.64\textwidth]{img_1194_small2.jpg}
%  \end{center}
%  \caption{In addition to verbally instructing participants to avoid
%    looking into the projector lens, we have installed warning signs
%    near each projector at eye level, in a clearly visible position.
%\label{FIGURE_warning_signs}
%}
%\end{figure}


%We made a concerted effort to obtain safety documentation on the
%intensity and spectral distribution of the light from this specific
%bulb directly from the manufacturer and distributor.  We contacted
%both Optoma and Phillips in December 2008 asking for the Spectral
%Power Distribution (SPD) of the equipment we have purchased from
%Optoma (the bulb is apparently manufactured by Phillips).  Both
%companies replied within a few days and we were told that the
%information was not available.  A second request to Phillips for the
%same information was made on May 4th 2009.  We called 1-800-937-5483
%and were told that ``UHP'' in the name of the lamp indicated that the
%bulb was an OEM part, thus the data was not available.  She suggested
%that we contact Optoma for the data.  She did give us the number of
%the OEM division of Phillips (1-866-915-5886).  From the OEM division,
%we were transferred to Technical Assistance, who spent about 30
%minutes searching for the specifications, but they were ultimately
%unable to locate the bulb (or any 180W projector lamp) in their
%equipment catalog.
%%
%%They gave us the
%%number of a specialty lamp division at Phillips (1-732-563-3339) and
%%...
%A second request to Optoma was made through their website on May 4th,
%2009.  The auto-reply email said we would here from a technical
%specialist within 24-48 hours, but we have not heard a response.  

%Since we were not able to obtain the information directly from these
%companies, we were forced to make these measurements and safety
%calculations ourselves.  

%%Using the SPD data and the table of the ``Spectral Weighting Functions
%%for Assessing Retinal Hazards from Broad-Band Optical Sources'' the
%%Blue Light Hazard Exposure Limits can be calculated [ANSI/IESNA
%%  RP-27.1-05].  

%%We attempted to borrow a spectroradiometer, equipment that can measure
%%the SPD, to directly measure the quantity and distribution of light
%%from these devices during normal operation.  Unfortunately,
%%Dr. Mariana Figueiro in the RPI Lighting Research Center said her lab
%%did not have equipment we could borrow.  Similarly, Dr. E. Fred
%%Schubert of the Future Chips Constellation said his lab did not have
%%any equipment capable of making these measurements.


%\section{Blue Light Hazard Calculation Detail}

%To evaluate the retinal photochemical hazard posed to study
%participants by light emitted by the six projectors used in the
%system, we calculated permissible time limits for exposure during an
%eight-hour period in accordance with ANSI/IESNA RP-27.1-05,
%specifically section 4.3.2, ``Retinal Blue Light Hazard Exposure
%Limit.''  Two calculations were performed: exposure limits for the
%system as it is intended to be used (study participants looking at
%matte white surfaces illuminated by up to six projectors), and
%accidental direct viewing of the lens of one projector from within the
%beam of that projector.

%During the intended use of the system, study participants will view
%matte-surface card-stock and foam-core board illuminated by up to six
%projectors.  For purposes of hazard evaluation, we assume 100\%
%reflectivity (measured at 95\%) and Lambertian reflectance
%characteristics.  We assume all six projectors are illuminating the
%table in a direct-incidence configuration, which is a worst-case
%situation, although not physically possible with the system
%(projectors are rigidly mounted in opposing directions).  Moreover, we
%assume that all six projectors produce the same spectral irradiance
%at the table surface, and take this irradiance to be the maximum
%measured over all six projectors (output varies due to bulb
%manufacturing tolerances and degradation over bulb lifetime).

%\begin{figure}[t]
%  \begin{center}
%    \includegraphics[width=0.6\textwidth]{table_radiance.pdf}
%  \end{center}
%  \caption{Measured worst-case radiance from table surfaces.
%\label{FIGURE_worst_case_radiance}
%}
%\end{figure}

%We measured the relative spectral output of the projectors with an
%Ocean Optics spectroradiometer and calibrated the measurements at the
%table surface to spectral irradiance ($W/m^2\cdot nm$) using a
%calibrated luxmeter with accurate CIE $V_m(\lambda)$ characteristics
%(Figure~\ref{FIGURE_worst_case_radiance}).  We evaluated the
%blue-light hazard function according to the following equation
%\cite{ANSI}:


%\[L_B = \sum_{300}^{700}L_{\lambda}\cdot B(\lambda)\cdot \Delta \lambda\]

%Given the worst-case assumptions presented above, we calculated a
%blue-light hazard weighted radiance reflected from the table of
%$1.73\times10^{-4}$ $W/cm^{-2}\cdot sr^{-1}$.  This value is less than
%2\% of the maximum allowed for time exceeding $10^4$ seconds (2.77h)
%\cite{ANSI}, a period substantially longer than the maximum 1.5 hour
%study participants will be allowed to use the system.  We conclude
%that there is no retinal photochemical injury hazard from the intended
%use of the system.

%Due to the design of the system, it is possible for study participants
%to accidentally look directly into the projector.  To evaluate this
%hazard, measurements were taken of the spectral radiance of the
%projectors at a distance of 30cm from the projector lens, which is the
%closest the eyes of a 6'-tall individual can get (measurements
%obtained with projectors off). At this distance the blue light hazard
%weighted radiance is $L_B = $ 43.2 $W/cm^{-2}\cdot sr^{-1}$. According
%to the formula \cite{ANSI}:

%\[t(max) = \frac{100}{L_B},\]

%\noindent
%the maximum permissible exposure is 2.3 seconds.  Since the perceived
%brightness of the projectors at this distance is comparable to that of
%direct noon sunlight, it is doubtful that study participants would
%accidentally stare directly into the projectors for extended periods
%of time, since we expect that this would be uncomfortable, and the
%natural reaction would be to look away and step out of the projector
%beam.  To mitigate the risk presented by possible exposure several
%steps will be taken.  First, study participants will be instructed
%both verbally and in writing not to look directly into the projectors
%since this may cause eye damage.  Additionally, warning signs have
%been posted (Figure~\ref{FIGURE_warning_signs}) in a clearly visible
%location near each projector warning participants of possible eye
%damage and instructing them not to look directly into the projector
%lenses.  Finally, during the study the person running the study will
%carefully monitor the participant and if the participant's gaze is
%drawn away from the table for an extended period, they will be
%encouraged to step completely away from the SAR frame to prevent
%accidentally gazing into the projector lens.

%As a verification of our measurement and calibration procedures, we
%ran identical calculations for noon sunlight and obtained maximum
%per-day exposure limits within 10\% of published results
%\cite{Okuno08}.








%\begin{thebibliography}{widest entry}
%  \bibitem{Okuno08} Okuno, Tsutomu, ``Hazards of Solar Blue Light,'' Applied Optics, V. 47,
%    No 16. June 1, 2008.
%  \bibitem{ANSI} ANSI/IES RP-27.1-05, ``Photobiological Safety for Lamps and Lamp Systems -
% General Requirements,'', Illuminating Engineering Society, 2005.
%\bibitem{dolce}
%Dolce, Andrew, 
%``Multi-User Interactions for Spatially Augmented Reality Games''
%Masters Thesis, Rensselaer Polytechnic Institute, 
%Department of Computer Science,
%June 2011.
%\url{http://www.cs.rpi.edu/graphics/theses/andrew_dolce_MS_May2011.pdf}
%\end{thebibliography}



%\newpage

%\pagestyle{empty}

%\begin{center}
%{\Large {
%Institutional Review Board \\ \vspace{0.1in}
%Rensselaer Polytechnic Institute} }\\
%\ \\
%{\large Informed Consent Form}
%\end{center}

%\noindent
%I understand that Barbara Cutler, who is a professor of Computer
%Science at Rensselaer Polytechnic Institute, would like me to use a
%new gaming interface and answer a short questionnaire as part of the
%research project on a new Spatially Augmented Reality (SAR) system for
%education and entertainment applied to games.  I understand that she
%will be making her best possible effort to guarantee me every possible
%protection, including the following:

%\begin{enumerate}

%\item I am 18 years or older.

%\item
%I am under no obligation to be participate in the study or to be
%  complete the questionnaire if I do not wish to do so.

%\item 
%I am not obligated to perform any of the game play exercises or
%  answer any of the questions.  I may decline to answer any or all of
%  the questions, and I may terminate the study at any point, without
%  giving any reason.

%\item 
%Participants for this study will be compensated for their time in the
%form of a gift certificate at the rate of \$10 per hour.  This
%compensation is not contingent upon the subject completing the entire
%study and will be prorated if the subject withdraws.

%\item I will be identified by a randomly assigned ID number that is
%  used only for this study.  All recordings and game state files will be
%  labeled with this ID.  All information and data relating to the user
%  study will be protected to secure confidentiality.  All electronic
%  files will be stored on password protected computers.  All paper
%  forms will be stored in a locked office.  The correspondence
%  between the ID number and my name will be recorded by Barbara Cutler
%  and be accessible only by her.  This correspondence will be
%  destroyed once analysis of the data is complete, within 1 year after
%  participation in the study.

%\item An overhead camera will capture still images of the game state
%  (without the hands or bodies of the players).  The game play session
%  may be recorded with a simple video camera (focusing on the table
%  surface, capturing only the players hands, not their faces or
%  bodies).  The players voices will also be recorded.  The audio
%  recordings will not be used, but quotations from the recordings may
%  be extracted for possible publication.  The video/audio recording
%  will be destroyed within 1 year after participation in the study.\\
%\noindent
%\hspace*{0.3in} \rule{0.3in}{1pt} I agree to be video and audio taped. \\
%\hspace*{0.3in} \rule{0.3in}{1pt} I do not agree to be video and audio taped.

%\item If there is anything that I do not wish to have quoted, or any
%  game state files that I do not want made public, I may say at any
%  point during or after the study what I wish to have kept off the
%  record and it will not be quoted or used in a publication.

%\item I understand that if Barbara Cutler decides to use any portions
%  of my answers to the questionnaire or any examples of my game play
%  or verbal comments during game play in subsequent publications, that
%  she will send me a copy of the portions of the questionnaire and any
%  game play, including any quotations and paraphrases that she decides
%  to use, for my editing and written approval.  I will have the right
%  to edit the material and I access to the final publication.  She
%  will only use the material that I have approved and the use of all
%  material will be anonymous.  I may also change my mind at any point
%  up to and including the review of any quotations and paraphrases and
%  game play that might be used.

%\end{enumerate}

%\vspace{0.3in}

%\begin{center}(continued on next page)\end{center}

%\newpage

%\begin{center}(continued from prev page)\end{center}

%\vspace{0.3in}

%\begin{enumerate}
%\setcounter{enumi}{8}

%\item The basic camera-projection Spatially Augmented Reality (SAR)
%  setup has been described to me and I have been warned not to look
%  directly at the projector lenses.  Standing close to the projector
%  (30cm) and looking directly into the projector bulb for 2 seconds or
%  longer may cause permanent eye damage.  I have been offered an
%  optional simple visor to wear during the study protect against
%  projector light from above.  The visor is not required for safe use
%  of the system.

%\end{enumerate}

%\vspace{0.5in}

%\noindent
%\rule{2.6in}{1pt}~~~~\rule{2.6in}{1pt}\hfill \rule{1in}{1pt}\\
%\hspace*{0.7in}Name of Participant 
%\hfill Signature \hspace{1.3in} 
%Date \hspace{0.3in}

%\vspace{0.15in}
%\noindent
%For further information contact:

%%\vspace{0.05in}
%\noindent
%Barbara Cutler, Department of Computer Science, MRC 330B, cutler@cs.rpi.edu,\\
%110 8th Street, Troy NY 12180; phone: (518) 276 3274, fax: (518) 276 2529.

%%\vspace{0.05in}
%\noindent
%Institutional Review Board, Rensselaer Polytechnic Institute, CII 9015, \\
%110 8th Street, Troy, New York, 12180; phone: (518) 276-4873, fax: (518) 276-4002.
%\vspace{-0.8in}

%\newpage

%\noindent
%{\bf {\Large Sample ARmy Game User Study Procedure and Instructions}}

%\vspace{0.2in}
%\noindent
%After receiving the system introduction and safety safety
%instructions, the basic rules of the game are described to the
%participants:

%\begin{enumerate}

%\item The platform, ramp, and wall modules are arranged on the table
%  to create an interesting terrain.  (The players agree on the
%  arrangement).

%\item The two players place their army units on the table.  Certain
%  rules for starting locations may apply (e.g., all army units must be
%  placed on that player's ``half'' of the table).

%\item Movement phases will alternate between the two players.  Combat
%  phases occur after each movement phase.  First the ``red'' player
%  moves.  Then we have a combat phase.  Then the ``green'' player
%  moves.  Then we have a combat phase.  Repeat until the game ends or
%  time limit for this game expires.

%  \item Each player will have a specified amount of time to consider
%    and finalize the movement of any or all units (e.g., 1 minute).
%    Each unit may move no more than a pre-specified distance (e.g., 3
%    inches), and may not ``jump'' onto or off of a platform.  The unit
%    must utilize available ramps to climb onto or off of platforms.
%    Units may not move through walls, etc.

%  \item After the player has moved any or all of their units, the
%    computer checks the current game state for validity.  If there is
%    an error, the system pauses while the player corrects the
%    problem.

%  \item At the beginning of combat, all opposing army units with a
%    line of sight between each other are determined.  During
%    ``combat'' each unit with a line of sight to the other unit
%    ``fires'' at the opposing unit and with some probability of
%    success (e.g., 25\%) disables the opposing unit.  All disabled
%    units are removed from the remainder of the game.

%\item If all of a players army units are removed from play, that
%  player loses and the game is over.

%\end{enumerate}

%\vspace{1.0in}


%\noindent
%Once the game rules are explained two or more games will be conducted.
%For example:

%\begin{enumerate}

%\item ($\sim$10 minutes) First, play a game without any computer
%  assistance or projector visualization augmentation.  Simply using
%  the physical modules and the game rules as described, the
%  participants will be asked to play a short game.  Rulers,
%  straightedges, and dice will be available to assist game play.

%\item ($\sim$10 minutes) Next, play the game with computer
%  management of game state (time limits, combat probabilities) and
%  projector visualization (realistic textures, movement circles, line
%  of sight visualization).

%\item Further games will explore variations on the data visualization
%  style, or sequencing of the games with and without augmentation will
%  be tested.

%\end{enumerate}


%\newpage

%\noindent
%{\bf {\Large Sample Post-Design Questionnaire}}

%\vspace{0.1in}

%\begin{enumerate}

%\item
%Were any aspects of game play {\em without} computer assistance and
%projector augmentation tedious, error-prone, subjective, vague, or
%underspecified?  Describe.

%\item
%Was the computer game management efficient, effective, accurate,
%and/or helpful?  Describe. 

%\item 
%Compare the computer game management/enforcement of the rules to the
%non-computerized game play.

%\item
%Was the computer visualization of movement circles and combat lines
%effective, accurate, and/or helpful?  Describe.

%\item
%Was the texturing of the surface geometry interesting?  Did it make
%the game playing experience more fun?  Did it make the game playing
%experience more immersive or engaging?

%\item
%What additional features would you like to see added to the system?

%\item 
%What other games do you think could be built using this system?  What
%other entertainment or education applications might benefit from this
%system?

\end{enumerate}


\end{document}
